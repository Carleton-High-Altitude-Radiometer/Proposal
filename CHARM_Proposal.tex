%%%%%%%%%%%%%%%%%%%%%%%%%%%%%%%%%%%%%%%%%%%%%%%%%%%%%%%%%%%%%%%%%%%%%
% LaTeX Template: Project Titlepage Modified (v 0.1) by rcx
%
% Original Source: http://www.howtotex.com
% Date: February 2014
% 
% This is a title page template which be used for articles & reports.
% 
% This is the modified version of the original Latex template from
% aforementioned website.
% 
%%%%%%%%%%%%%%%%%%%%%%%%%%%%%%%%%%%%%%%%%%%%%%%%%%%%%%%%%%%%%%%%%%%%%%

\documentclass[12pt]{article}
\usepackage[a4paper]{geometry}
\usepackage[myheadings]{fullpage}
\usepackage{fancyhdr}
\usepackage{lastpage}
\usepackage{graphicx, wrapfig, subcaption, setspace, booktabs}
\usepackage[T1]{fontenc}
\usepackage[font=small, labelfont=bf]{caption}
\usepackage{fourier}
\usepackage[protrusion=true, expansion=true]{microtype}
\usepackage[english]{babel}
\usepackage{sectsty}
\usepackage{url, lipsum}

\newcommand{\HRule}[1]{\rule{\linewidth}{#1}}
\onehalfspacing
\setcounter{tocdepth}{5}
\setcounter{secnumdepth}{5}

%-------------------------------------------------------------------------------
% HEADER & FOOTER
%-------------------------------------------------------------------------------
\pagestyle{fancy}
\fancyhf{}
\setlength\headheight{15pt}
\fancyhead[L]{CHARM}
\fancyhead[R]{Carleton University}
\fancyfoot[R]{Page \thepage\ of \pageref{LastPage}}
%-------------------------------------------------------------------------------
% TITLE PAGE
%-------------------------------------------------------------------------------

\begin{document}
\bibliographystyle{ieeetr}

\title{ \normalsize \textsc{}
		\\ [2.0cm]
		\HRule{0.5pt} \\
		\LARGE \textbf{{Carleton High Altitude Radiometer}}\\
		\large \textbf{{Project Proposal}}\\
		\large \textbf{{Canadian Stratospheric Balloon Experiment Design Challenge}}
		\HRule{2pt} \\ [0.5cm]
		\normalsize \today \vspace*{2\baselineskip}\\
		\includegraphics[scale=0.6]{Figures/CHARM.png}	\vspace*{2\baselineskip} \\
		\textsc{
		David Bascelli \quad Team Lead \\
		Jacob Booth \quad Sensor Lead \\
		}}
		
\date{}
	
\author{
		Carleton University \\
		}
	


\maketitle

\newpage

\tableofcontents
\newpage

\listoffigures
\newpage

\listoftables
\newpage

%-------------------------------------------------------------------------------
% Section title formatting
\sectionfont{\scshape}
%-------------------------------------------------------------------------------

%-------------------------------------------------------------------------------
% BODY
%-------------------------------------------------------------------------------

\section{Executive Summary}
\newpage

\section{Proposal}
\subsection{Scientific Objectives}

Microwave remote sensing has been a common payload on earth observation satellites since the early days of space-flight. In as early as 1962, a radiometer on-board the Mariner 2 mission measured the surface temperature of Venus. 1968 Saw the first space-born earth observation radiometer on-board the  Cosmos 243 satellite, which measured atmospheric water vapour and global ice cover. Many different radiometer configurations can make a wide array of different geological, biological, and climate measurements. Our scientific objective is to make low cost measurements of soil moisture content using a balloon born microwave radiometer. Soil moisture measurements are crucial in predicting local weather conditions and monitoring climate change. Incorporating soil moisture measurements into weather and climate models allows for more accurate medium term weather forecasts and can also give clues about future droughts, crop yields, and water resource management. Currently, most radiometric data comes from space-born radiometers, such as those on the SMAP or SMOS satellites. To achieve high resolution and accuracy, these space-born radiometers utilize cryogenic components, complex phased array or synthetic aperture technologies, and require large and very directional antennas. Our belief is that similar measurements could be performed from a high altitude balloon at significantly reduced cost. Balloon born radiometers even have some advantages to space-born radiometers, including reduced antenna directionality requirements and reduced atmospheric effects. 
 
\subsection{Experiment Design}
\subsubsection{Outline}

\subsubsection{Radiometer Block Diagram}
David
\subsubsection{Attitude Determination}
Jacob
\subsubsection{Experimental Procedures}
David
\subsubsection{Resources}
David
\subsubsection{Technical Risk Assessment}
Jacob
\begin{enumerate}
\item Human
\cite{omar_el-kassaby_abdelghaffar_2017}
\item Technical and Environmental
\end{enumerate}
\subsection{Management}
David
\subsubsection{Team Structure}
\subsubsection{Project Time-line}
\subsubsection{Budget}
\subsubsection{Managerial Risk Assessment}

\subsection{Outreach}
Jacob
\subsubsection{Public Outreach}
\subsubsection{Academic Outreach}

\section{Conclusion}

%-------------------------------------------------------------------------------
% REFERENCES
%-------------------------------------------------------------------------------
\newpage
\section{References}

\bibliography{library}

\section{Appendix}


\end{document}
